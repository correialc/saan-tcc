\section{Pré-processamento}

O pré-processamento é a etapa responsável por ajustar a representação dos dados a um formato adequado aos algoritmos de aprendizado de máquina. Esta etapa contempla uma série de tarefas como remoção de pontuação, exclusão de \textit{stopwords}, tokenização,  \textit{stemming} e vetorização.

A classe \textbf{Preprocessamento} (preprocessamento.py) realiza as tarefas necessárias ao pré-processamento dos dados. Seguindo a mesma lógica das etapas anteriores, o objeto de dados (instância da classe Dados) traz os dados provenientes da etapa de Limpeza (atributo \textbf{limp}) e armazena os resultados do pré-processamento no atributo \textbf{prep}.  

\subsection{Remoção de Pontuação}

Caracteres de pontuação geralmente não contribuem com o resultado de algoritmos de classificação por serem muito repetitivos e guardarem pouca ou nenhuma informação específica de uma classe. Para a remoção dos caracteres de pontuação foi utilizada a coleção de carecteres \textit{punctuation}, parte da biblioteca padrão do Python e acessível através do módulo \textit{string}. Para remover os caracteres de pontuação foi necessário somente ler os textos dos segmentos e eliminar os caracteres do texto que pertenceciam à coleção \textit{punctuation}.

\subsection{Exclusão de \textit{Stopwords}}

\textit{Stopwords} são termos comuns em uma linguagem, como artigos e preposições, que não trazem contribuição para o resultado de algoritmos de classificação, de forma similar ao que ocorre com os caracteres de pontuação. Tanto no caso da pontuação, quanto no caso das \textit{stopwords}, a remoção dos termos permite uma redução do vocabulário, simplificando a representação do texto e permitindo uma atuação mais específica dos algoritmos de classificação. Para exclusão das \textit{stopwords} foi utilizada a coleção \textit{nltk.corpus.stopwords.words} em português do NLTK (\textit{Natural Language Toolkit})\footnote{https://www.nltk.org/}, uma biblioteca com diversas funções para o processamento de linguagem natural.

\subsection{Tokenização}

Tokenização é a tarefa de subdividir o texto em seus elementos fundamentais ou \textit{tokens}. A tokenização permite a identificação do conjunto de termos únicos que compõem o texto (o vocabulário), permitindo a aplicação de funções específicas termo-a-termo (como as funções de \textit{stemming}). Para realização da tokenização foi utilizada a biblioteca \textit{RegexpTokenizer} do NLTK que executa uma tokenização baseada em expressões regulares. Foram testadas outras bibliotecas de tokenização do NLTK como \textit{WhitespaceTokenizer} e \textit{WordPunctTokenizer}, mas ambas apresentaram resultados inferiores de tokenização.

\subsection{Stemming}

\textit{Stemming} é a redução de um palavra ao seu radical, ou seja, a exclusão de sufixos presentes em variações de uma mesma palavra. As palavras menino, menina, meninos,  meninas, menininho, menininha e meninice, após um processo de \textit{stemming}, seriam reduzidas ao radical \textbf{menin}. Esta eliminação das variações permite tratar diferentes palavras como um único \textit{token}, reduzindo o vocabulário e consequentemente o espaço de estados das caraterísticas de entrada dos modelos de classificação. Para realização do processo de \textit{stemming} foi utilizado o \textit{SnowballStemmer} do NLTK.