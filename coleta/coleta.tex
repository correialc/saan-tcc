\section{Coleta de Dados}

Para a realização deste trabalho foi utilizado um conjunto de dados contendo 260.488 segmentos pertencentes a 20.821 atos do sistema Normas\footnote{Nem todos os atos do sistema Normas são de domínio público, mas todos os atos utilizados nesta pesquisa são públicos e podem ser acessados através da Internet, tanto pela Imprensa Nacional quanto pelo sistema Normas.}, relativos ao período de 01/01/2018 a 31/12/2020. Os dados estão em formato *.csv (arquivo extracacao-segmentos-atos.csv) e seguem a estrutura descrita na tabela \ref{tab:estrutura-conjunto-dados}.

\begin{table}[h!] 
\caption{Estrutura do conjunto de dados}
\label{tab:estrutura-conjunto-dados}
	\begin{center} 
		\begin{tabular}{|l|l|l|} 
			\hline ATRIBUTO & DESCRIÇÃO & TIPO \\
			\hline
			\hline id\textunderscore ato & Identificador do ato & Quantitativo Discreto \\ 
			\hline data\textunderscore  pub & Data de publicação do ato & Categórico Nominal \\ 
			\hline tipo\textunderscore  ato & Tipo do ato & Categórico Nominal \\
			\hline id\textunderscore seg & Identificador do segmento & Quantitativo Discreto \\
			\hline tipo\textunderscore seg & Tipo do segmento & Categórico Nominal \\
			\hline txt\textunderscore seg & Texto do segmento & Categórico Nominal \\
			\hline
		\end{tabular}
	\end{center}
	\fdp
\end{table}

Os atributos id\textunderscore ato, data\textunderscore  pub, tipo\textunderscore  ato e id\textunderscore seg foram utilizados somente para a análise exploratória e posterior limpeza dos dados. Os atributos relevantes para a criação do modelo de classificação foram txt\textunderscore seg (que representa o texto do segmento a ser classificado) e tipo\textunderscore seg (a classe ou \textit{label} do segmento).

A classe CargaDados (carga\textunderscore dados.py) é responsável por realizar a o processo de carga a partir do conjunto de dados de origem (extracacao-segmentos-atos.csv). A importação de dados foi realizada através do comando \textit{read\textunderscore csv} do Pandas. Os dados importados foram armazenados em uma instância da classe \textbf{Dados} (dados.py). Essa instância, denominada daqui por diante \textbf{objeto de dados}, foi utilizada ao longo de várias etapas do fluxo de aprendizado de máquina para armazenar dados temporários, atuando como um repositório de dados compartilhado.

O código a seguir apresenta um exemplo de execução da etapa de carga a partir do notebook fluxo-ml.ipynb: 

\begin{lstlisting}
	dados = Dados()
	cg = CargaDados()
	cg.executar(dados)
	12:34:32 - Carregando dados de segmentos...
	12:34:32 - 206488 registros carregados.
\end{lstlisting}

Como a etapa de carga é a primeira a ser executada, é preciso inicialmente instanciar o objeto de dados. A seguir é criada uma instância da classe CargaDados, chamando o método ``executar'' e passando o objeto de dados como parâmetro. A carga de dados é então realizada e a quantidade de segmentos carregados é exibida. 


