\chapter{Coleta de Dados}

Para a realização deste trabalho foi utilizado um conjunto de dados contendo 260.488 segmentos pertencentes a 20.821 atos do sistema Normas\footnote{Nem todos os atos do sistema Normas são de domínio público, mas todos os atos utilizados nesta pesquisa são públicos e podem ser acessados através da Internet, tanto pela Imprensa Nacional quanto pelo sistema Normas.}, relativos ao período de 01/01/2018 a 31/12/2020. Os dados estão em formato *.CSV e seguem a estrutura descrita na tabela \ref{tab:estrutura-conjunto-dados}.

\begin{table}[h!] 
\caption{Estrutura do Conjunto de Dados}
\label{tab:estrutura-conjunto-dados}
	\begin{center} 
		\begin{tabular}{|l|l|l|} 
			\hline ATRIBUTO & DESCRIÇÃO & TIPO \\
			\hline
			\hline id\textunderscore ato & Identificador do ato & Quantitavivo Discreto \\ 
			\hline data\textunderscore  pub & Data de publicação do ato & Categórico Nominal \\ 
			\hline tipo\textunderscore  ato & Tipo do ato & Categórico Nominal \\
			\hline id\textunderscore seg & Identifiador do segmento & Quantitavivo Discreto \\
			\hline tipo\textunderscore seg & Tipo do segmento & Categórico Nominal \\
			\hline txt\textunderscore seg & Texto do segmento & Categórico Nominal \\
			\hline
		\end{tabular} 
	\end{center}
\end{table}

Os atributos id\textunderscore ato, data\textunderscore  pub, tipo\textunderscore  ato e id\textunderscore seg foram utilizados somente para a análise exploratória e posterior limpeza dos dados. Os atributos relevantes para a criação do modelo de classificação foram txt\textunderscore seg (que representa o texto do segmento a ser classificado) e tipo\textunderscore seg (a classe ou \textit{label} do segmento).  