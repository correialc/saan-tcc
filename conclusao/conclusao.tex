\section{Conclusão}

Utilizando técnicas de classificação de texto baseadas no Processamento de Linguagem Natural foi possível desenvolver três modelos de classificação de segmentos, um para cada tipo de ato analisado (ADE, SC e PORT). Os atos pertencentes a esses três tipos representam juntos 97,37\% de todos os atos não sigilosos publicados no sistema Normas entre 2018 e 2020.

Os três modelos de classificação de segmentos desenvolvidos alcançaram resultados acima de 90\% em todas as métricas avaliadas, com destaque para os atos dos tipos ADE e SC que atingiram resultados acima de 97\%, evidenciando uma contribuição significativa deste trabalho para a automatização do processo de inclusão de atos da RFB.

Grande parte da codificação realizada ao longo deste trabalho foi desenvolvida de forma a ser reutilizada em ambientes produtivos, o que evidencia uma contribuição complementar desta pesquisa.

Adicionalmente, esta pesquisa abre caminho para uma série de investigações mencionadas aqui como sugestões para trabalhos futuros: 
\begin{itemize}
	\item Avaliação do desempenho de outros algoritmos de classificação no mesmo cenário avaliado aqui;
	\item Utilização de técnicas de \textit{word embeddings} e \textit{deep learning} visando capturar o contexto dos termos que formam os segmentos dos atos;
	\item Investigação aprofundada dos segmentos do tipo ``Não Identificado'' que foram a maior fonte de inconsistência nas classificações realizadas.
\end{itemize}

Com base nos dados apresentados e nos resultados obtidos é possível afirmar que este trabalho atingiu o objetivo proposto.


%Figura criada para resolver um bug na numeracao de figuras e tabelas
\begin{figure}
\label{fig:fake}
\end{figure}