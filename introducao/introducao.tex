\chapter{Introdução}

\section{Contextualização}

A publicação de atos normativos é uma etapa fundamental do processo de gestão pública, pois formaliza e divulga para a sociedade as decisões do Governo Federal. A Receita Federal do Brasil\footnote{Disponível em \url{https://www.gov.br/receitafederal/pt-br}. Acesso em 06-jun-2021.} (RFB) disponibiliza atos normativos através do sistema Normas\footnote{Disponível em \url{http://normas.receita.fazenda.gov.br}. Acesso em 06-jun-2021.}, acessível publicamente através da Internet. Os atos são inicialmente publicados no Diário Oficial da União (DOU) através da Imprensa Nacional\footnote{Disponível em \url{https://www.in.gov.br/acesso-a-informacao/dados-abertos/base-de-dados}. Acesso em 06-jun-2021.} e incluídos manualmente no sistema Normas por uma equipe da RFB.

Um ato normativo é composto por segmentos, trechos que possuem um significado próprio e formatação específica definidos pelo decreto XYZ que estabelece as regras para elaboração de atos normativos. A classificação dos segmentos é uma etapa fundamental para o processo de inclusão de atos, mas atualmente é realizada de forma manual. No período entre 2013 e 2020 foram incluídos aproximadamente XX atos normativos compostos por cerca de YY segmentos, evidenciando um esforço grande de classificação manual.

\section{O problema proposto}



