\section{Limpeza de Dados}

A etapa de limpeza de dados é responsável por excluir dados com valores inválidos ou pouco representativos. Os principais pontos identificados na análise exploratória de dados foram a exclusão de atos (e seus segmentos) que não pertençam aos tipo ADE, SC e PORT, a remoção de segmentos do tipo Anexo, a eliminação de valores ausentes e a exclusão de caracteres de \textit{escape} e \textit{tags} HTML.

A classe \textbf{LimpezaDados} (limpeza\textunderscore dados.py) é responsável por buscar os dados provenientes da etapa de coleta (já carregados no objeto de dados), realizar as tarefas de limpeza e gravar os dados de volta no objeto de dados. Os atributos \textbf{orig} e \textbf{limp} definidos na classe Dados armazenam  respectivamente os dados da coleta e os dados após a limpeza. O código a seguir apresenta um exemplo de execução da etapa de limpeza a partir do notebook fluxo-ml.ipynb:

\begin{lstlisting}
	lp = LimpezaDados(dados)
	lp.executar(dados, 'ADE')
	12:57:52 - Excluindo segmentos dos atos que nao sao ADE...
	12:57:52 - 91448 segmentos de atos nao ADE excluidos.
	12:57:52 - Restaram 115040 segmentos de atos ADE.
	12:57:52 - Removendo segmentos nao representativos...
	12:57:52 - 6827 segmentos nao representativos excluidos.
	12:57:52 - Restaram 108213 segmentos representativos.
	12:57:52 - Removendo segmentos nulos...
	12:57:52 - 69 segmentos nulos excluidos.
	12:57:52 - Restaram 108144 segmentos nao nulos.
	12:57:52 - Removendo tags HTML...
	12:57:52 - Removendo caracteres de escape HTML...
	12:57:52 - Limpeza de dados concluida.
\end{lstlisting}

Na etapa de limpeza não é mais necessário instanciar o objeto de dados pois a etapa de carga já realizou essa tarefa. Dessa forma, o primeiro passo da limpeza é criar uma instância da classe LimpezaDados, chamando então o método ``executar'' e passando o objeto de dados como parâmetro. 

O tipo de ato precisa ser fornecido como parâmetro adicional. Isso ocorre porque, durante a análise exploratória de dados, ficou evidente que alguns tipos de segmento não existem para alguns tipos de ato ou ocorrem com uma frequência muito baixa (seção \ref{sec:dist-atos-segmentos}). Como o tipo de segmento é a variável alvo para o modelo de classificação, remover os tipos de segmentos menos frequentes ou ausentes para um tipo de ato reduz a quantidade de classes do modelo, tornando-o mais específico para as classes mais frequentes. O tipo de ato é previamente conhecido (ou pode ser facilmente deduzido), logo, fornecer o tipo do ato como parâmetro não representa um obstáculo. 

Esses fatores acabaram direcionando a estratégia de classificação para a criação de modelos específicos de acordo com o tipo de ato. Dessa forma, os Atos Declaratórios Executivos (ADE) tiveram um modelo de classificação específico, assim como as Soluções de Consulta (SC) e também as Portarias (PORT). Cada um desses modelos teve como domínio da variável alvo o conjunto de tipos de segmento pertencentes aquele tipo de ato. Nesse cenário, para os atos do tipo ADE, por exemplo, foram considerados como variável alvo os tipos de segmento Artigo, Ementa, Fecho, Inciso e Não Identificado enquanto para os atos do tipo SC foram contemplados os segmentos do tipo Ementa, Fecho e Não Identificado. O limite de corte utilizado foi um valor menor do que 0,01 (1\%), calculado usando a fórmula a seguir.

\[ corte = \frac{qtd\_segmentos}{total\_segmentos} \]

\begin{quote}
Sendo \( corte \) o limite de corte dos segmentos, \( qtd\_segmentos \) a quantidade de segmentos encontrados (\textbf{de um determinado tipo de segmento}) para o tipo de ato informado e \( total\_segmentos \) a quantidade de segmentos encontrados (\textbf{de todos os tipos de segmento}) para o tipo de ato informado.
\end{quote}

Isso acabou trazendo uma tarefa adicional para a etapa de limpeza: remover os atos e segmentos não relacionados ao tipo de ato informado. Assim, a partir do parâmetro tipo\textunderscore ato, a classe de limpeza realiza duas tarefas: 
\begin{alineas}
	\item Excluir os atos (e seus segmentos) de tipos diferentes do indicado pelo parâmetro tipo\textunderscore ato (exemplo: se tipo\textunderscore ato='SC', excluir todos os atos 'ADE', 'PORT' e 'OUTROS');
	\item Remover os segmentos que pertencem a tipos de segmento não relacionados (por ausência ou baixa frequência) ao tipo de ato indicado pelo parâmetro tipo\textunderscore ato (exemplo: se tipo\textunderscore ato='SC', remover todos os segmentos que não sejam Ementa, Fecho ou Não Identificado).
\end{alineas}

No exemplo de código mencionado anteriormente, o parâmetro tipo\textunderscore ato foi igual a 'ADE', logo, foram removidos 91.448 segmentos de atos que não eram ADE e 6.827 segmentos dos tipos Autor, Capítulo, Item, Parágrafo, Seção, Subseção e Título, não representativos para os atos do tipo ADE, conforme informado na tabela \ref{tab:segmentos-por-tipo}. 