\section{Análise de Resultados}

Foram realizadas 24 combinações de treinamento e teste: dois algoritmos de classificação (\textit{LinearSVC} e \textit{MultinomialNB}); dois algoritmos de classificação com combinação de classes (\textit{OneVsOne} e \textit{OnevsRest}); treinamento com e sem validação cruzada; treinamento de três tipos de ato distintos (ADE, SC e PORT). As tabelas \ref{tab:resultados-ade}, \ref{tab:resultados-sc} e \ref{tab:resultados-port} apresentam os resultados para os atos dos tipos ADE, SC e PORT respectivamente.

Com base nos resultados encontrados, é possível concluir que os modelos treinados com o algoritmo \textit{LinearSVC} apresentaram melhor desempenho em todas métricas para todos os tipos tipos de ato se comparados aos modelos treinados com o algoritmo \textit{MultinomialNB}. A diferença entre a utilização dos algoritmos de combinação de classes \textit{OneVsOne} e \textit{OnevsRest} foi pequena. Como critério de comparação entre as métricas o XXX foi priorizado ...

Outra constatação importante foi um resultado muito inferior para os atos do tipo PORT. Uma possível explicação para esse é o fato das portarias possuírem segmentos do tipo parágrafo que ...

\begin{table}[h]
\caption{Resultados para os atos do tipo ADE}
\label{tab:resultados-ade}
	\begin{center}
	\begin{tabular}{lrrrr}
		\toprule
		{} &  Acurácia &  Precisão &  Revocacão &      F1 \\
		\midrule
		LinearSVC-1x1        &    0.9839 &    0.9776 &     0.9806 &  0.9790 \\
		LinearSVC-1xR        &    0.9813 &    0.9764 &     0.9754 &  0.9759 \\
		MultinomialNB-1x1    &    0.9282 &    0.9245 &     0.9246 &  0.9234 \\
		MultinomialNB-1xR    &    0.9297 &    0.9282 &     0.9258 &  0.9259 \\
		LinearSVC-1x1-CV     &    0.9739 &    0.9665 &     0.9648 &  0.9653 \\
		LinearSVC-1xR-CV     &    0.9720 &    0.9630 &     0.9649 &  0.9636 \\
		MultinomialNB-1x1-CV &    0.9158 &    0.9052 &     0.9122 &  0.9071 \\
		MultinomialNB-1xR-CV &    0.9157 &    0.9083 &     0.9129 &  0.9090 \\
		\bottomrule
	\end{tabular}
	\end{center}		
\end{table}

\begin{table}[h]
\caption{Resultados para os atos do tipo SC}
\label{tab:resultados-sc}
	\begin{center}
	\begin{tabular}{lrrrr}
		\toprule
		{} &  Acuracia &  Precisão &  Revocação &      F1 \\
		\midrule
		LinearSVC-1x1        &    0.9920 &    0.9946 &     0.9882 &  0.9914 \\
		LinearSVC-1xR        &    0.9906 &    0.9933 &     0.9872 &  0.9902 \\
		MultinomialNB-1x1    &    0.8296 &    0.9086 &     0.8079 &  0.8272 \\
		MultinomialNB-1xR    &    0.8291 &    0.9043 &     0.8088 &  0.8278 \\
		LinearSVC-1x1-CV     &    0.9777 &    0.9857 &     0.9641 &  0.9740 \\
		LinearSVC-1xR-CV     &    0.9780 &    0.9860 &     0.9619 &  0.9730 \\
		MultinomialNB-1x1-CV &    0.8161 &    0.8630 &     0.7955 &  0.8018 \\
		MultinomialNB-1xR-CV &    0.8129 &    0.8507 &     0.7950 &  0.7968 \\
		\bottomrule
	\end{tabular}
	\end{center}		
\end{table}

\begin{table}[h]
\caption{Resultados para os atos do tipo PORT}
\label{tab:resultados-port}
	\begin{center}
	\begin{tabular}{lrrrr}
		\toprule
		{} &  acuracia &  precisao &  revocacao &      f1 \\
		\midrule
		LinearSVC-1x1        &    0.8961 &    0.9131 &     0.8706 &  0.8894 \\
		LinearSVC-1xR        &    0.8953 &    0.9075 &     0.8783 &  0.8914 \\
		MultinomialNB-1x1    &    0.7399 &    0.8414 &     0.6516 &  0.6870 \\
		MultinomialNB-1xR    &    0.7511 &    0.8370 &     0.6881 &  0.7212 \\
		LinearSVC-1x1-CV     &    0.8413 &    0.8692 &     0.8111 &  0.8321 \\
		LinearSVC-1xR-CV     &    0.8387 &    0.8538 &     0.8141 &  0.8285 \\
		MultinomialNB-1x1-CV &    0.6890 &    0.7810 &     0.6097 &  0.6338 \\
		MultinomialNB-1xR-CV &    0.6999 &    0.7850 &     0.6410 &  0.6619 \\
		\bottomrule
	\end{tabular}
	\end{center}		
\end{table}

isuahdashiudh isahdihasiudhasiu hdiuashi dhasihd iashdi hasidh ashdi ashdihas iudhsaihi hasdih ashdia shdih ihdish hidshid hisahdi hhisdhhi sdisah idsiah iashdi hsaihd iashdi hasidh iashdias idahsh idsiah hdiasdi hhsaih dhiusahid uhsaiuhd iushidu isuhdhi ushiduhsa uihdiusa hhdiush iusdahiu hsdajdsa.