
\documentclass[
	article,
	12pt,				
	oneside,
	a4paper,
	english,
	brazil
]{abntex2}


% Configuracoes de fonte
\usepackage{helvet}	
\renewcommand{\familydefault}{\sfdefault}		
\usepackage[utf8]{inputenc}
\usepackage[T1]{fontenc}		

\renewcommand{\ABNTEXchapterfontsize}{\LARGE} % Tamanho dos títulos dos capítulos

% Pacotes fundamentais 
\usepackage{indentfirst}		
\usepackage{color}				
\usepackage{graphicx}			
\usepackage{microtype} 
\usepackage{pucmg}	% Customizacoes para os padroes PUC-MG
% ---

% Pacotes de citações
\usepackage[brazilian,hyperpageref]{backref}	 
\usepackage[alf,abnt-emphasize=bf]{abntex2cite}	% Citações padrão ABNT

% Informações de dados para CAPA e FOLHA DE ROSTO
\titulo{\Large{CLASSIFICAÇÃO DE SEGMENTOS EM ATOS NORMATIVOS ATRAVÉS DE PROCESSAMENTO DE LINGUAGEM NATURAL}}
\autor{Leandro Coelho Correia}
\local{Salvador}
\data{2021}
\instituicao{Pontifícia Universidade Católica de Minas Gerais}
\departamento{Núcleo de Educação à Distância}
\filiacao{Pós-graduação em Inteligência Artificial e Aprendizado de Máquina}
\tipotrabalho{Relatório técnico}
\preambulo{Trabalho de Conclusão de Curso apresentado ao Curso de Especialização em Inteligência Artificial e Aprendizado de Máquina como requisito parcial à obtenção do título de especialista.}
% ---

% informações do PDF
\makeatletter
\hypersetup{
     	%pagebackref=true,
		pdftitle={\@title}, 
		pdfauthor={\@author},
    	pdfsubject={\imprimirpreambulo},
	    pdfcreator={LaTeX with abnTeX2},
		pdfkeywords={abnt}{latex}{abntex}{abntex2}{relatório técnico}, 
		bookmarksdepth=4
}
\makeatother
% --- 

% Espaçamentos
\setlength\afterchapskip{12pt} % Após o título dos capítulos
\setlength\beforesecskip{12pt} % Antes do título das seções
\setlength\aftersecskip{12pt} 	% Após o título das seções
\setlength{\parindent}{1.3cm} 	% Identação da primeira linha do parágrafo
\setlength{\parskip}{0.2cm}  	% Espaçamento entre um parágrafo e outro

% Compila o índice
\makeindex


% Início do documento
\begin{document}

\selectlanguage{brazil}

% Retira espaço extra obsoleto entre as frases.
\frenchspacing 

% ----------------------------------------------------------
% ELEMENTOS PRÉ-TEXTUAIS
% ----------------------------------------------------------
\pretextual

\imprimircapa
\imprimirfolhaderosto*

% inserir lista de ilustrações
\pdfbookmark[0]{\listfigurename}{lof}
\listoffigures*
\cleardoublepage
% ---

% inserir lista de tabelas
\pdfbookmark[0]{\listtablename}{lot}
\listoftables*
\cleardoublepage
% ---

% ---
% inserir o sumario
% ---
\pdfbookmark[0]{\contentsname}{toc}
\tableofcontents*
\cleardoublepage
% ---

% ------------------
% ELEMENTOS TEXTUAIS
% ------------------
\textual

\begingroup
\let\clearpage\relax
\chapter{Introdução}

\section{Contextualização}
...

\section{O problema proposto}
...
\chapter{Coleta de Dados}

Para a realização deste trabalho foi utilizado um conjunto de dados contendo XXX segmentos pertencentes a YYY atos públicos do sistema Normas, relativos ao período de 01/01/2018 a 31/12/2020. Os dados estão em formato *.CSV e seguem a estrutura descrita na tabela \ref{tab:estrutura-coleta}.

    
\endgroup

\newpage

% ----------------------------------------------------------
% ELEMENTOS PÓS-TEXTUAIS
% ----------------------------------------------------------
\postextual

% Referências bibliográficas
\bibliography{referencias}

\end{document}
