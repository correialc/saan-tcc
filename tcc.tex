
\documentclass[
	article,
	12pt,				
	oneside,
	a4paper,
	english,
	brazil
]{abntex2}


% Configuracoes de fonte
\usepackage{helvet}	
\renewcommand{\familydefault}{\sfdefault}		
\usepackage[utf8]{inputenc}
\usepackage[T1]{fontenc}		

\renewcommand{\ABNTEXchapterfontsize}{\LARGE} % Tamanho dos títulos dos capítulos

% Pacotes fundamentais 
\usepackage{indentfirst}		
\usepackage{color}				
\usepackage{graphicx}			
\usepackage{microtype}
\usepackage{listings}
\usepackage{pucmg}	% Customizacoes para os padroes PUC-MG
% ---

% Pacotes de citações
\usepackage[alf,abnt-emphasize=bf]{abntex2cite}	% Citações padrão ABNT

% Definições de estilo para a listagem de código
\lstloadlanguages{Python}

\definecolor{codegreen}{rgb}{0,0.6,0}
\definecolor{codegray}{rgb}{0.5,0.5,0.5}
\definecolor{codepurple}{rgb}{0.58,0,0.82}
\definecolor{backcolour}{rgb}{0.95,0.95,0.92}

\lstdefinestyle{estiloCodigo}{
    language=Python,
    backgroundcolor=\color{backcolour},   
    commentstyle=\color{codegreen},
    keywordstyle=\color{magenta},
    numberstyle=\tiny\color{codegray},
    stringstyle=\color{codepurple},
    basicstyle=\ttfamily,
    identifierstyle=\color{blue},
    breakatwhitespace=false,         
    breaklines=true,                 
    keepspaces=true,                 
    numbers=none,       
    numbersep=5pt,                  
    showspaces=false,                
    showstringspaces=false,
    showtabs=false,                  
    tabsize=2,
}

\lstset{style=estiloCodigo}
% ---

% Informações de dados para CAPA e FOLHA DE ROSTO
\titulo{\Large{CLASSIFICAÇÃO DE SEGMENTOS EM ATOS NORMATIVOS ATRAVÉS DE PROCESSAMENTO DE LINGUAGEM NATURAL}}
\autor{Leandro Coelho Correia}
\local{Salvador}
\data{2021}
\instituicao{Pontifícia Universidade Católica de Minas Gerais}
\departamento{Núcleo de Educação à Distância}
\filiacao{Pós-graduação em Inteligência Artificial e Aprendizado de Máquina}
\tipotrabalho{Relatório técnico}
\preambulo{Trabalho de Conclusão de Curso apresentado ao Curso de Especialização em Inteligência Artificial e Aprendizado de Máquina como requisito parcial à obtenção do título de especialista.}
% ---

% informações do PDF
\makeatletter
\hypersetup{
		pdftitle={\@title}, 
		pdfauthor={\@author},
    	pdfsubject={\imprimirpreambulo},
	    pdfcreator={LaTeX with abnTeX2},
		pdfkeywords={abnt}{latex}{abntex}{abntex2}{relatório técnico}, 
		bookmarksdepth=4
}
\makeatother
% --- 

% Espaçamentos
\setlength\afterchapskip{12pt} % Após o título dos capítulos
\setlength\beforesecskip{12pt} % Antes do título das seções
\setlength\aftersecskip{12pt} 	% Após o título das seções
\setlength{\parindent}{1.3cm} 	% Identação da primeira linha do parágrafo
\setlength{\parskip}{0.2cm}  	% Espaçamento entre um parágrafo e outro

% Compila o índice
\makeindex


% Início do documento
\begin{document}

\selectlanguage{brazil}

% Retira espaço extra obsoleto entre as frases.
\frenchspacing 

% ----------------------------------------------------------
% ELEMENTOS PRÉ-TEXTUAIS
% ----------------------------------------------------------
\pretextual

\imprimircapa
\imprimirfolhaderosto*

% inserir lista de ilustrações
\pdfbookmark[0]{\listfigurename}{lof}
\listoffigures*
\cleardoublepage
% ---

% inserir lista de tabelas
\pdfbookmark[0]{\listtablename}{lot}
\listoftables*
\cleardoublepage
% ---

% ---
% inserir o sumario
% ---
\pdfbookmark[0]{\contentsname}{toc}
\tableofcontents*
\cleardoublepage
% ---

% ------------------
% ELEMENTOS TEXTUAIS
% ------------------
\textual

\begingroup
\let\clearpage\relax
\chapter{Introdução}

\section{Contextualização}
...

\section{O problema proposto}
...
\chapter{Coleta de Dados}

Para a realização deste trabalho foi utilizado um conjunto de dados contendo XXX segmentos pertencentes a YYY atos públicos do sistema Normas, relativos ao período de 01/01/2018 a 31/12/2020. Os dados estão em formato *.CSV e seguem a estrutura descrita na tabela \ref{tab:estrutura-coleta}.

    
\section{Análise Exploratória}

A Análise Exploratória de Dados (AED) é uma etapa do fluxo de aprendizado de máquina que permite realizar uma exploração dos dados baseada em técnicas da Estatística Descritiva. Para essa etapa da pesquisa foi utilizado o notebook \textbf{analise-exploratoria.ipynb}. Por se tratar de uma atividade interativa, a utilização de um Jupyter Notebook se mostrou mais eficiente do que a utilização direta de um script Python.

\subsection{Valores Ausentes ou Inválidos}

A primeira investigação realizada foi a busca de valores valores ausentes. Foram encontrados 1.900 segmentos com valores ausentes para o atributo txt\textunderscore seg. Nos demais atributos não havia valores ausentes. A tabela \ref{tab:valores-ausentes} apresenta a quantidade de segmentos com valores ausentes por tipo de segmento.

\begin{table}[h] 
\caption{Valores ausentes por tipo de segmento}
\label{tab:valores-ausentes}
	\begin{center} 
		\begin{tabular}{|l|r|} 
			\hline TIPO DE SEGMENTO & VALORES AUSENTES \\
			\hline
			\hline Anexo & 1.778 \\
			\hline Não Identificado & 95 \\			
			\hline Fecho & 21 \\
			\hline Ementa & 2 \\			
			\hline Artigo & 2 \\
			\hline Título & 1 \\
			\hline Alínea & 1 \\			
			\hline
		\end{tabular}
	\end{center}
	\fdp
\end{table} 

Os segmentos do tipo Anexo representam 93,57\% dos segmentos com valores ausentes. Como esse tipo de segmento representa arquivos binários associados aos atos, todos os segmentos desse tipo podem ser descartados. Além dos 1.778 apresentados na tabela \ref{tab:valores-ausentes}, existem outros 5.771 segmentos do tipo anexo com texto desprezível (ponto, vírgula, nome de arquivo, etc) totalizando 7.549 segmentos selecionados para exclusão durante a etapa de limpeza de dados. Os demais segmentos com valores ausentes representam 0.06\% do total (122/198.939) e foram também selecionados para exclusão.

Além dos valores ausentes, em diversos tipos de segmento foram identificados caracteres inválidos que precisaram ser removidos no processo de limpeza de dados, especialmente os caracteres de \textit{escape} e \textit{tags} HTML (exemplos: <br/>, \&ccedil, \&atilde).

\subsection{Distribuição de Atos e Segmentos \label{sec:dist-atos-segmentos}}

Um aspecto que chamou a atenção na exploração dos dados foi a predominância de Atos Declaratórios Executivos (ADE), representando 71,79\% (14.948 dos 20.821 atos analisados), seguido pelas Soluções de Consulta (SC) com 14,98\% e Portarias (PORT) com 10,60\%. Os demais tipos de ato representam juntos 2.63\% do total. A figura \ref{fig:atos-por-tipo-ato} evidencia essa distribuição.

\begin{figure}[h]
	\caption{Quantidade de atos por tipo de ato}
	\center
	\label{fig:atos-por-tipo-ato}
	\includegraphics[scale=1.9]{exploratoria/atos-por-tipo-ato.png}
	\fdp
\end{figure}

Outro aspecto importante analisado foi a distribuição dos segmentos por tipo de ato e tipo de segmento. Na tabela \ref{tab:segmentos-por-tipo} é possível perceber que dos 17 tipos possíveis de segmento, somente 13 estão presentes em atos dos tipos ADE, SC e PORT. Além disso, nem todos os 13 tipos de segmento estão presentes nos 3 tipos de ato e alguns tipos de segmento são pouco frequentes.

\begin{table}[h] 
\caption{Quantidade de segmentos por tipo de ato e tipo de segmento}
\label{tab:segmentos-por-tipo}
	\begin{center} 

		\begin{tabular}{rrrr}
		\toprule
		TIPO DE ATO &      ADE    &   PORT &      SC \\
		TIPO DE SEGMENTO          &        &         \\
		\midrule
		Alínea           &    780 &  2.265 &      87 \\
		Artigo           & 37.921 & 11.578 &       3 \\
		Autor            &      3 &      4 &       0 \\
		Capítulo         &      5 &    381 &       0 \\
		Ementa           & 14.948 &  2.207 &   3.119 \\
		Fecho            & 14.853 &  2.289 &     919 \\
		Inciso           &  4.134 & 17.399 &       0 \\
		Item             &    398 &    388 &      10 \\
		Não Identificado & 36.288 & 11.726 &   6.609 \\
		Parágrafo        &    861 &  5.894 &       0 \\
		Seção            &      3 &    219 &       0 \\
		SubSeção         &      0 &     18 &       0 \\
		Título           &    545 &    540 &       0 \\
		\bottomrule
	\end{tabular}
	\end{center}
	\fdp
\end{table} 

Os segmentos do tipo ``Não Identificado'' representam um ponto de atenção para a classificação dos segmentos. Por ser o tipo padrão de segmento do sistema Normas, sempre que ocorre falha humana por omissão na classificação manual, o segmento fica classificado como ``Não Identificado''. Apesar disso, essa é uma classe válida de segmento amplamente utilizada para representar trechos do ato que não precisam de formatação específica. Por estar entre os tipos mais frequentes de segmento e ser um tipo válido, os segmentos desse tipo não podem ser removidos do escopo da classificação, mas a presença de segmentos com classificação omissa pode prejudicar os resultados dos modelos.

\subsection{Identificação de Padrões}

Ao longo da análise exploratória foram identificados, em alguns tipos de segmento, padrões de texto  que podem ser utilizados como heurísticas baseadas em expressões regulares. Se essas heurísticas apresentarem resultados melhores do que o modelo de classificação, será possível retirar alguns tipos de segmento do escopo da classificação, deixando o modelo mais específico para as demais classes. A seguir são apresentados os padrões encontrados:

\begin{alineas}
	\item Ementas podem ser iniciadas por um verbo ou pela expressão ``Assunto: ''. Exemplo: ``Cancela o registro especial para estabelecimento que realiza operação com papel[...]'';
	\item Artigos iniciam com a abreviação ``Art.'' seguida de um numeral ordinal. Exemplo: ``Art. 2\textsuperscript{o} O detalhamento do motivo da exclusão poderá ser obtido[...]'';
	\item Incisos são iniciados com um numeral romano seguido de hífem. Exemplo: ``IV – Fundamento legal para reconhecimento do direito[...]'';
	\item Alíneas iniciam com letras (minúsculas ou maiúsculas) seguidas de parênteses. Exemplo: ``f) desempenhar as tarefas inerentes ao sistema de progressão funcional[...]'';
	\item Parágrafos podem ser iniciados com a expressão ``Parágrafo único'' ou com o símbolo ``§'' seguido de um numeral ordinal. Exemplo: ``§ 2\textsuperscript{o} Compete à chefia imediata a gestão da frequência dos seus servidores[...]'';
	\item Fechos são iniciados com nomes próprios, seguidos ou não pelo cargo.
\end{alineas}

 \subsection{Conclusões da Análise Exploratória}
 
 Ao final da análise exploratória de dados foi possível elencar as seguintes conclusões que servirão de base para as etapas seguintes do fluxo de aprendizado de máquina:

\begin{alineas}
	\item Os segmentos do tipo Anexo podem ser removidos do conjunto de dados por não representarem conteúdo textual;
	\item Os segmentos com valores ausentes são pouco representativos (0,06\% do total de segmentos) e podem ser removidos do conjunto de dados;
	\item Será necessário tratar caracteres inválidos na etapa de limpeza;
	\item Os tipos de ato ADE, SC e PORT representam 97,37\% do total de atos. O demais tipos de ato juntos são bem menos frequentes e os atos desses tipos podem ser removidos do conjunto de dados;	 
	\item Os tipos de segmento (variável alvo da classificação) variam de um tipo de ato para outro e isso precisa ser considerado nas etapas seguintes;
	\item Existem padrões nos textos de alguns tipos de segmento que permitem a avaliação de heurísticas alternativas à utilização de algoritmos de classificação.
\end{alineas}

\section{Limpeza de Dados}

A etapa de limpeza de dados é responsável por excluir dados com valores inválidos ou pouco representativos. Os principais pontos identificados na análise exploratória de dados foram a exclusão de atos que não pertençam aos tipo ADE, SC e PORT, a remoção de segmentos do tipo Anexo, a eliminação de valores ausentes e a exclusão de caracteres de \textit{escape} e \textit{tags} HTML.

A classe LimpezaDados (limpeza\textunderscore dados.py) é responsável por buscar os dados provenientes da etapa de coleta em uma instância da classe Dados, realizar as tarefas de limpeza e gravar os dados de volta na classe Dados. Os atributos \textbf{orig} e \textbf{limp} da classe Dados armazenam  respectivamente os dados da coleta e os dados após a limpeza.
\section{Pré-processamento}

O pré-processamento é a etapa responsável por ajustar a representação dos dados a um formato adequado aos algoritmos de aprendizado de máquina. Esta etapa contempla uma série de tarefas como remoção de pontuação, exclusão de \textit{stopwords}, tokenização,  \textit{steeming} e vetorização.

A classe \textbf{Preprocessamento} (preprocessamento.py) realiza as tarefas necessárias ao pré-processamento dos dados. Seguindo a mesma lógica das etapas anteriores, o objeto de dados (instância da classe Dados) traz os dados provenientes da etapa de Limpeza (atributo \textbf{limp}) e armazena os resultados do pré-processamento no atributo \textbf{prep}.  

\subsection{Remoção de Pontuação}

Caracteres de pontuação geralmente não contribuem com o resultado de algoritmos de classificação por serem muito repetitivos e guardarem pouca ou nenhuma informação específica de uma classe. Para a remoção dos caracteres de pontuação foi utilizada a coleção de carecteres \textit{punctuation}, parte da biblioteca padrão do Python e acessível através do módulo \textit{string}. Para remover os caracteres de pontuação foi necessário somente ler os textos dos segmentos e eliminar os caracteres do texto que pertenceciam à coleção \textit{punctuation}.

\subsection{Exclusão de \textit{Stopwords}}

\textit{Stopwords} são termos comuns em uma linguagem, como artigos e preposições, que não trazem contribuição para o resultado de algoritmos de classificação, de forma similar ao que ocorre com os caracteres de pontuação. Tanto no caso da pontuação, quanto no caso das \textit{stopwords}, a remoção dos termos permite uma redução do vocabulário, simplificando a representação do texto e permitindo uma atuação mais específica dos algoritmos de classificação. Para exclusão das \textit{stopwords} foi utilizada a coleção \textit{nltk.corpus.stopwords.words} em português do NLTK (\textit{Natural Language Toolkit})\footnote{https://www.nltk.org/}, uma biblioteca com diversas funções para o processamento de linguagem natural.

\subsection{Tokenização}

Tokenização é a tarefa de subdividir o texto em seus elementos fundamentais ou \textit{tokens}. A tokenização permite a identificação do conjunto de termos únicos que compõem o texto (o vocabulário), permitindo a aplicação de funções específicas termo-a-termo (como as funções de \textit{steeming}). Para realização da tokenização foi utilizada a biblioteca \textit{RegexpTokenizer} do NLTK que executa uma tokenização baseada em expressões regulares. Foram testados outras bibliotecas de tokenização do NLTK como o \textit{WhitespaceTokenizer} e o \textit{WordPunctTokenizer}, mas ambos apresentaram resultados inferiores de tokenização.

\subsection{Stemming}

\textit{Stemming} é a redução de um palavra ao seu radical, ou seja, a exclusão de sufixos presentes em variações de uma mesma palavra. As palavras menino, menina, meninos,  meninas, menininho, menininha e meninice, após um processo de \textit{stemming}, seriam reduzidas ao radical \textbf{menin}. Esta eliminação das variações permite tratar diferentes palavras como um único \textit{token}, reduzindo o vocabulário e consequentemente o espaço de estados das caraterísticas de entrada dos modelos de classificação. Para realização do processo de \textit{stemming} foi utilizado o \textit{SnowballStemmer} do NLTK.
\section{Extração de Características}

A extração de características ou vetorização converte o texto para uma representação matemática que possa ser utilizada como entrada para os algoritmos de aprendizado de máquina. Apesar de poder ser considerada como parte da etapa de pré-processamento, foi tratada, aqui no texto e no código-fonte, de forma separada por possuir um conjunto maior de detalhes de implementação.

\subsection{Preparação dos Conjuntos de Dados}

A vetorização trata o objeto de dados de forma diferente das etapas anteriores. A classe \textbf{ExtracaoCaracteristicas} (extracao\textunderscore caracteristicas.py) busca os resultados do pré-processamento (atributo \textbf{prep}) e realiza a separação do dados em um conjunto de treino (\textbf{Xtr} e \textbf{Ytr}) e outro de teste (\textbf{Xte} e \textbf{Yte}). Essa separação é fundamental para que os modelos sejam treinados com um conjunto de dados e testados em um conjunto de dados de teste desconhecido do processo de treinamento. Foi utilizada uma separação 80/20 (80\% dos dados para treinamento e 20\% dos dados para teste). 

Além da separação de treino e teste, foram criados mais dois atributos na classe Dados (\textbf{X} e \textbf{y}) para armazenar o conjunto completo de dados. Esses atributos foram utilizados para validação cruzada. A validação cruzada gera vários subconjuntos de treino e teste permitindo que o treinamento dos modelos de aprendizado de máquina seja realizado com base em diversas configurações dos dados, reduzindo as chances de que uma configuração específica apresente um resultado muito melhor ou muito pior do que as demais.  

\subsection{Vetorização TF-IDF}

Após a preparação dos conjuntos de dados, foi utilizado o módulo \textit{TfIdfVectorizer} da biblioteca \textit{SciKit-Learn} para realização da vetorização. TF-IDF (\textit{Term Frequency - Inverse Document Frequency}) é uma representação vetorial do texto que leva em consideração a relação entre a frequência de cada termo em um documento (TF) e sua raridade em todo o conjunto de documentos, medida pela frequência inversa de documento (IDF). Dessa forma, cada documento vetorizado via TF-IDF se transforma em um vetor de características e o   \textit{corpus}\footnote{O conjunto de todos os documentos é conhecido em PLN como \textit{corpus}.} pode ser representado matematicamente como o conjuntos desses vetores de características \cite{DSNegocios2016}.

 
\endgroup

\newpage

% ----------------------------------------------------------
% ELEMENTOS PÓS-TEXTUAIS
% ----------------------------------------------------------
\postextual

% Referências bibliográficas
\bibliography{referencias}

\end{document}
