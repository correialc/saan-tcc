
\documentclass[
	12pt,				
	oneside,
	a4paper,
	english,
	brazil,
	]{abntex2}


% Pacotes fundamentais 
\usepackage{helvet}			
\usepackage[T1]{fontenc}		
\usepackage[utf8]{inputenc}		
\usepackage{indentfirst}		
\usepackage{color}				
\usepackage{graphicx}			
\usepackage{microtype} 			
% ---
	

% Pacotes de citações
\usepackage[brazilian,hyperpageref]{backref}	 
\usepackage[alf]{abntex2cite}	% Citações padrão ABNT

% Informações de dados para CAPA e FOLHA DE ROSTO
\titulo{\Large{CLASSIFICAÇÃO DE SEGMENTOS EM ATOS NORMATIVOS ATRAVÉS DE PROCESSAMENTO DE LINGUAGEM NATURAL}}
\autor{Leandro Coelho Correia}
\local{Salvador}
\data{2021}
\instituicao{%
  Pontifícia Universidade Católica de Minas Gerais
  \par
  Núcleo de Educação à Distância
  \par
  Pós-graduação Lato Sensu em Inteligência Artificial e Aprendizado de Máquina}
\tipotrabalho{Relatório técnico}
\preambulo{Trabalho de Conclusão de Curso apresentado ao Curso de Especialização em Inteligência Artificial e Aprendizado de Máquina como requisito parcial à obtenção do título de especialista.}
% ---

% informações do PDF
\makeatletter
\hypersetup{
     	%pagebackref=true,
		pdftitle={\@title}, 
		pdfauthor={\@author},
    	pdfsubject={\imprimirpreambulo},
	    pdfcreator={LaTeX with abnTeX2},
		pdfkeywords={abnt}{latex}{abntex}{abntex2}{relatório técnico}, 
		bookmarksdepth=4
}
\makeatother
% --- 

% O tamanho do parágrafo é dado por:
\setlength{\parindent}{1.3cm}

% Controle do espaçamento entre um parágrafo e outro:
\setlength{\parskip}{0.2cm}  % tente também \onelineskip

% Compila o índice
\makeindex


% Início do documento
\begin{document}


\selectlanguage{brazil}

% Retira espaço extra obsoleto entre as frases.
\frenchspacing 

% ----------------------------------------------------------
% ELEMENTOS PRÉ-TEXTUAIS
% ----------------------------------------------------------
\pretextual

\imprimircapa
\imprimirfolhaderosto*

% inserir lista de ilustrações
\pdfbookmark[0]{\listfigurename}{lof}
\listoffigures*
\cleardoublepage
% ---

% inserir lista de tabelas
\pdfbookmark[0]{\listtablename}{lot}
\listoftables*
\cleardoublepage
% ---

% ---
% inserir o sumario
% ---
\pdfbookmark[0]{\contentsname}{toc}
\tableofcontents*
\cleardoublepage
% ---

% ----------------------------------------------------------
% ELEMENTOS TEXTUAIS
% ----------------------------------------------------------
\textual

% ------------------
% ELEMENTOS TEXTUAIS
% ------------------
\textual

\chapter{Introdução}
\section{Contextualização}
...
\section{O problema proposto}
...

\chapter{Coleta de Dados}
...

\chapter{Processamento/Tratamento dos Dados}
...

\chapter{Análise e Exploração dos Dados}
...

\chapter{Criação de Modelos de Machine Learning}
...

\chapter{Apresentação dos Resultados}
...

\chapter{Conclusão}
...

% ----------------------------------------------------------
% ELEMENTOS PÓS-TEXTUAIS
% ----------------------------------------------------------
\postextual

% Referências bibliográficas
%\bibliography{referencias.bib}

\end{document}
