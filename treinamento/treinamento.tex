\section{Treinamento e Teste}

Para a etapa de treinamento e teste foram criadas duas classes: \textbf{Treinamento} (treinamento.py) e \textbf{Classificador} (classificador.py). Em Classificador foram criados alguns  métodos relacionados ao processo de execução da validação cruzada (quando solicitado), treinamento, cálculo de métricas e apresentação da matriz de confusão para um algoritmo específico de classificação. A classe Treinamento é responsável por armazenar uma coleção de instâncias de classificadores, executando o treinamento e o cálculo de métricas para todos eles.

Como se trata de um problema de classificação multi-classe, foram treinados os algoritmos \textit{LinearSVC} (uma implementação de \textit{Support Vector Machine} para classificação) e \textit{MultinomialNB} (uma implementação de classificador bayesiano) em combinação com o algoritmo \textit{OneVsRestClassifier}\footnote{No algoritmo \textit{LinearSVC} essa combinação é implícita, ou seja, o próprio \textit{LinearSVC} implementa internamente uma abordagem \textit{One vs Rest} quando a quantidade de classes é superior a 2.} que permite realizar diferentes combinações de classes em problemas de classificação multi-classe. Adicionalmente, todas as combinações de modelos foram realizadas com e sem validação cruzada, a título de comparação (por ser mais rigoroso, o resultado com validação cruzada foi utilizado como referência). Todos os algoritmos utilizados fazem parte da biblioteca \textit{SciKit-Learn}. O código a seguir apresenta uma das combinações de algoritmos de classificação com validação cruzada realizadas:

\begin{lstlisting}
# Multinomial Naive Bayes 
t = Treinamento()
estimadorNB_1xR_CV = OneVsRestClassifier(MultinomialNB())
t.adicionar_modelo(
	Classificador('MultinomialNB-1xR-CV', estimadorNB_1xR_CV))
t.treinar_modelos(dados, cv=5)
\end{lstlisting}

Foram adotadas quatro métricas para avaliação dos resultados: 
\begin{alineas}
	\item Acurácia: de todas as classificações, quantas o modelo classificou corretamente;
	\item Precisão: de todas as classificações positivas do modelo, quantas classificações positivas estavam corretas;
	\item Revocação: de todas as classificações positivas esperadas, quantas classificações positivas estavam corretas;
	\item  F1-Score: média harmônica entre a precisão e a revocação.
\end{alineas}

Todas as métricas utilizadas estão disponíveis através do módulo \textit{metrics} do \textit{SciKit-Learn}. Assim como outros dados ao longo das etapas do fluxo de aprendizado de máquina, as métricas também foram armazenadas no objeto de dados para análise posterior. 