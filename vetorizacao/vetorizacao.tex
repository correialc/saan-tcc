\section{Extração de Características}

A extração de características ou vetorização converte o texto para uma representação matemática que possa ser utilizada como entrada para os algoritmos de aprendizado de máquina. Apesar de poder ser considerada como parte da etapa de pré-processamento, foi tratada, aqui no texto e no código-fonte, de forma separada por possuir um conjunto maior de detalhes de implementação.

\subsection{Preparação dos Conjuntos de Dados}

A vetorização trata o objeto de dados de forma diferente das etapas anteriores. A classe \textbf{ExtracaoCaracteristicas} (extracao\textunderscore caracteristicas.py) busca os resultados do pré-processamento (atributo \textbf{prep}) e realiza a separação dos dados em um conjunto de treino (\textbf{Xtr} e \textbf{Ytr}) e outro de teste (\textbf{Xte} e \textbf{Yte}). Essa separação é fundamental para que os modelos sejam treinados com um conjunto de dados e testados em um conjunto de dados de teste desconhecido do processo de treinamento. Foi utilizada uma separação 80/20 (80\% dos dados para treinamento e 20\% dos dados para teste). 

Além da separação de treino e teste, foram criados mais dois atributos na classe Dados (\textbf{X} e \textbf{y}) para armazenar o conjunto completo de dados. Esses atributos foram utilizados para validação cruzada. A validação cruzada gera vários subconjuntos de treino e teste permitindo que o treinamento dos modelos de aprendizado de máquina seja realizado com base em diversas configurações dos dados, reduzindo as chances de que uma configuração específica apresente um resultado muito melhor ou muito pior do que as demais.  

\subsection{Vetorização TF-IDF}

Após a preparação dos conjuntos de dados, foi utilizado o módulo \textit{TfIdfVectorizer} da biblioteca \textit{SciKit-Learn} para realização da vetorização. TF-IDF (\textit{Term Frequency - Inverse Document Frequency}) é uma representação vetorial do texto que leva em consideração a relação entre a frequência de cada termo em um documento (TF) e sua raridade em todo o conjunto de documentos, medida pela frequência inversa de documento (IDF). Dessa forma, cada documento vetorizado via TF-IDF se transforma em um vetor de características e o   \textit{corpus}\footnote{O conjunto de todos os documentos é conhecido em PLN como \textit{corpus}.} pode ser representado matematicamente como o conjuntos desses vetores de características \cite{DSNegocios2016}.

 